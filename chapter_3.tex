\chapter{需要用到的神经网络}
在\LaTeX 中排版数学公式需要$amsmath$宏包(已经包含在本模板中),对多行公式的排版提供了有力的支持。在实验报告的书写中,主要以$amsmath$宏包的内容为主,其余内容不做阐述。
\section{公式排版基础}
在\LaTeX 中书写数学公式时必须带有数学环境,数学环境内可以识别特殊的命令并且字体改变为数学字体,一般数学环境有两种,书写方式如下:
\begin{itemize}
\item \textbf{行内公式:} 行内公式是出现在文字陈述中间的数学公式,需要用双\$符号括起来,例如我们知道对于矩阵而言,乘法交换率是不成立的,也就是说\\$\forall A\neq B,\, \exists A\times B\neq B\times      A$(\verb|$A\times B\neq B\times      A$|),书写时除去命令后必须的空格外,其他的空格会被一概忽略,至于如何添加空白会在后面叙述。
\item \textbf{行间公式:} 行间公式是出现在文字陈述段落中间的数学公式,一般需要编号。当行间公式需要编号时可以使用equation环境,不需要编号时可以使用简单的方式编写行间公式:\verb|\[myEquation\]|
\end{itemize}

在数学环境内,不允许有多于的空格与空行,需要强制空格可以使用命令\verb|\,|、\verb|\quad|、\verb|\qquad|等,他们的产生的空白距离有所不同。其次数学环境中的所有字母都会被当做变量处理,采用数学字体显示,当需要在数学环境中输入公式时,可以使用命令\verb|\text{}|。
\section{排版数学公式}
在以往的实验报告中,数学公式都会使用word中的mathtype书写,在较高的版本中可以复制mathtype为\LaTeX 代码,但这种投机取巧的方法写出来的符号会非常的丑,并且速度不会比直接使用\LaTeX 书写快多少。下面,作者会首先叙述一部分必要的知识,其次的内容会以实例的方式展现,需与tex文档(源文档)配合学习。

\begin{enumerate}[1.]
\item 上标的表示方式为\verb|a^{2}|,显示结果为$a^{2}$,当上标内容单一时可以省略大括号,如\verb|a^2|也可以显示为$a^2$,当需要输入符号$\^$时输入\verb|\^|即可。
\item 下表的表示方式为\verb|a_{2}|,显示结果为$a_{2}$,当小标内容单一时也可以省略大括号,当需要输入符号$\_$时输入\verb|\_|即可。
\item 同时需要上下标时书写没有先后顺序,\verb|a^{x+y}_{x_1}|与\verb|a_{x_1}^{x+y}|结果都是$a_{x_1}^{x+y}$。
\item 对于巨运算符,如果直接书写\verb|\sum^{n}_{i=1}n!|会容易显示为$\sum^{n}_{i=1}n!$,而添加命令\verb|\limits|后,\verb|\sum\limits^{n}_{i=1}n!|则会显示为$\sum\limits^{n}_{i=1}n!$。
\item 书写分数的命令为\verb|\frac{text}{den}|,其中text为分子部分,den部分为分母部分,如\verb|\frac{1}{2}|会显示为$\frac{1}{2}$,如果觉得分数略小可以适当的使用命令\verb|\dfrac{text}{den}|
显示为$\dfrac{1}{2}$。
\item 导数直接使用单引号即可\verb|f' f'' f'''|显示为$f' f'' f'''$,常见的运算符号与巨运算符如\\\verb|+ - \times \div = \sum \prod \int|,分别为$+ - \times \div = \sum \prod \int$,更多的基础符号见amsmath宏包或lshort,下面也给出了支持所有的符号大全与手写符号识别的网址。
\LaTeX 支持的符号大全:
\url{http://mirrors.ctan.org/info/symbols/comprehensive/symbols-a4.pdf}

手写符号识别:
\url{http://detexify.kirelabs.org/classify.html}
\item 需要输入大括号时需输入\verb|\{\}|。
\end{enumerate}
下面会用一些函数、习题、定理、证明过程或是计算过程作为实例:

\textbf{Stolz 定理}:设$\{y_n\}$是严格单调增加的正无穷大量,且
\[
\lim\limits_{n \to \infty}\frac{x_n-x_{n-1}}{y_n-y_{n-1}}=a\quad (a\text{可以为有限量,}+\infty\text{与}-\infty)\text{,}
\]
则
\[
\lim\limits_{n \to \infty}\frac{x_n}{y_n}=a\text{。}
\]

求极限
\[
\lim\limits_{n \to \infty}\frac{1^k+2^k+\cdots+n^k}{n^{k+1}}(k\text{为正整数})\text{。}
\]

由已知,可得
\begin{equation}\label{equ1}
\lim\limits_{n \to \infty}\frac{a^n}{n!}=0\text{。}
\end{equation}

设函数
\[
f(x)=(\frac{x+\exp^{\frac{1}{x}}}{1+\exp^{\frac{4}{x}}}+\frac{\sin x}{|x|})\text{,}
\]
问当$x\to 0$时,$f(x)$的极限是否存在?

设$f(x)$在$[a,b]$上连续,且$f(x)>0$,证明
\begin{equation}
\frac{1}{b-a}\int_{a}^{b}\ln f(x)\text{d}x\leqslant \ln \left(\frac{1}{b-a}\int_{a}^{b}f(x)\text{d}x\right)\text{。}
\end{equation}

常用的希腊字符如下:
\[\alpha \beta \gamma \delta \varepsilon \zeta \theta \eta \mu \xi \pi \sigma \omega \phi\]
\section{多行公式的排版}
在书写报告时,时长会遇到多行排版,如矩阵、分段函数等等,在下文将介绍部分常用的环境,用于排版多行公式。

多行排版的环境使用方式大致类似,需要对齐的位置利用$\&$分割,行末需要使用\verb|\\|分割。
\subsection{align环境}
align环境会给环境内的每一行公式编号,去掉编号可以使用\verb|\notag|,使用方式如下:
\begin{minted}{LaTeX}
\begin{align}
a & = b + c \\
a & = b + c \\
x + y & = d + e \notag
\end{align}
\end{minted}
显示为
\begin{align}
a & = b + c \\
a & = b + c \\
x + y & = d + e \notag
\end{align}

align环境会在$\&$符号处对齐,多个$\&$会分段对齐。
\subsection{aligned环境}
与align环境不同,aligned环境会给公式整体一个编号,而不是每一行都有编号,同时需要equation环境套在外面。
\begin{minted}{LaTeX}
\begin{equation}
	\begin{aligned}
		a & = b + c \\
		x + y & = d + e 
	\end{aligned}
\end{equation}
\end{minted}
显示为
\begin{equation}
	\begin{aligned}
		a & = b + c \\
		x + y & = d + e 
	\end{aligned}
\end{equation}
split 环境和aligned 环境用法类似,也用于和equation 环境套用,区别是split 只能
将每行的一个公式分两栏,aligned 允许每行多个公式多栏。
\subsection{array环境}
array环境用于排版数学数组、矩阵等,数组可作为一个公式块,在外套用\verb|\left|、\verb|\right|等定界符。跟在环境名后的\verb|{cccc}|意为矩阵4列均居中(c代表居中、l代表左对齐、r代表右对齐,更仔细的将在表格排版处讲述),具体实例如下:
\begin{minted}{LaTeX}
\[ 
\mathbf{X} = \left(
\begin{array}{cccc}
x_{11} & x_{12} & \ldots & x_{1n}\\
x_{21} & x_{22} & \ldots & x_{2n}\\
\vdots & \vdots & \ddots & \vdots\\
x_{n1} & x_{n2} & \ldots & x_{nn}\\
\end{array} \right) 
\]
\end{minted}
显示为
\[ 
\mathbf{X} = \left(
\begin{array}{cccc}
x_{11} & x_{12} & \ldots & x_{1n}\\
x_{21} & x_{22} & \ldots & x_{2n}\\
\vdots & \vdots & \ddots & \vdots\\
x_{n1} & x_{n2} & \ldots & x_{nn}\\
\end{array} \right) 
\]
其中,\verb|\left(|、\verb|\right)|就是矩阵的定界符,小括号可以替换为\verb|[ {|等。
\subsection{case环境}
借用之前所述的定界符,使用单侧定界符可以用来书写分段函数,如$\textbf{Riemann}$函数可以书写为:
\begin{minted}{LaTeX}
\[
\mathbf{R}(x)=\left\{\begin{array}{ll}
\frac{1}{p},&x=\frac{q}{p}(p\in \mathbf{N}^{+},q\in \mathbf{Z}-\{0\},p,q\text{互质}),\\
1,&x=0,\\
0,&x\text{为无理数}
\end{array}\right.
\]
\end{minted}
显示为
\[
\mathbf{R}(x)=\left\{\begin{array}{ll}
\frac{1}{p},&x=\frac{q}{p}(p\in \mathbf{N}^{+},q\in \mathbf{Z}-\{0\},p,q\text{互质}),\\
1,&x=0,\\
0,&x\text{为无理数}
\end{array}\right.
\]
对于这类分段函数,可以使用更简单的cases环境来完成
\begin{minted}{LaTeX}
\[
\mathbf{R}(x)=\begin{cases}
\frac{1}{p},&x=\frac{q}{p}(p\in \mathbf{N}^{+},q\in \mathbf{Z}-\{0\},p,q\text{互质}),\\
1,&x=0,\\
0,&x\text{为无理数}
\end{cases}
\]
\end{minted}
显示为
\[
\mathbf{R}(x)=\begin{cases}
\frac{1}{p},&x=\frac{q}{p}(p\in \mathbf{N}^{+},q\in \mathbf{Z}-\{0\},p,q\text{互质}),\\
1,&x=0,\\
0,&x\text{为无理数}
\end{cases}
\]